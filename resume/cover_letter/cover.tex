\documentclass[11pt,a4paper,roman]{moderncv}
\usepackage{babel}

\moderncvstyle{classic}                            
\moderncvcolor{green}                              

\usepackage[utf8]{inputenc}                       

\usepackage[scale=0.75]{geometry}

% personal data
\name{}{Ryan Walker}
\address{Vancouver}{BC, Canada}{}
\phone[mobile]{650-796-8864}
\email{ryan.cjw@gmail.com}
\homepage{www.interruptlabs.ca}

%----------------------------------------------------------------------------------
%            content
%----------------------------------------------------------------------------------
\begin{document}
%-----       letter       ---------------------------------------------------------
% recipient data
\recipient{BCIT}{Hiring Department}
\date{\today}
\opening{To who it my concern,}
\closing{Thank you for your consideration.}
\makelettertitle

I'm a young passionate engineer that has been working in industry for almost 10 years. I have experience in electrical design, embedded firmware and embedded Linux systems. I like systems which require hard realtime deadlines and algorithm development. I started my career working for a prosthetic company, I designed robots for 3D scanning of patients limbs. The data was used to reconstruct artificial sockets, which were milled from foam and then vacuum formed on top of. I designed all the embedded electronics for the control of the scanners and milling machines. I've worked with industrial control systems, multi ton machinery used for processing lumber. Following this I worked for five years at a consulting company, I was exposed to a variety of embedded challenges. I started off building a IOT devices designed to monitor air quality. This device used lasers and a photodiodes, it counted the particles in the air and presented this data to the user to alert them if they may be in a dangerous situation. It used mqtt to push this data to an Amazon EC2 instance which updated an S3 bucked, I didn't design the app that took it from there. Following this I developed all the electronics in a lifesaving device. Vodasafe (https://vodasafe.ca/) featured a STM32F7 MCU connected to an LCD screen to help lifeguards locate a drowning victim. It featured a flyback SMPS that charges capacitors to 300V, the device used an H-Bridge to dump the energy into a ceramic load. After the energy was dissipated a TR switch connect the transducer to a LNA to then be sampled back into the MCU. This was effectively a handheld sonar device to save lives. I was contracted by large (think FANG) company to build a insect traps. These are currently being used to understand vector borne disease propagation through developing countries, the PCBs hosted high current motor controllers to control a robotic like apparatus. I've written real time firmware for medical devices that are currently being used in research labs, they use a microcontroller to precisely control the amount of current flowing through someone's brain. They use ADCs to sample back the response. They are  being used with patients who suffer from severe Parkinson's decease. 

Currently I'm working for and undisclosed company writing realtime firmware in bare metal and Linux/android.


I am very interested in the Mechatronics and Robotics faculty position as this was a program that enabled my career. I started the program immediately after high school and finished in two years, after which I found employment working in several different companies and fields. I worked towards my degree part-time at BCIT while working in the field until the end of 2019. At the start of 2020, I moved to Seattle, working on an engineering TN Visa writing firmware for one of the biggest tech companies in the world: Facebook. My path through working at several startups and a large tech firm give me very current industry knowledge, and skill that will greatly benefit the robotics program students. I have always had a strong desire to teach and am passionate about electronics and engineering. I work on open source projects and software on the side and have a popular instructional youtube channel which I hope outlines my desire to teach and my eLearning skills.


\makeletterclosing

\end{document}


%% end of file `template.tex'.
